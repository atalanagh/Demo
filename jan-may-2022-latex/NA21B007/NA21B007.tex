\documentclass{article}
\title{\textbf{Mass–energy equivalence}}
\date{June 2022}
\author{Anagh Atal NA21B007}

\usepackage{graphicx}


\begin{document}
  \maketitle
\large
\paragraph{} In physics, \textbf{Mass–energy equivalence} is the relationship between mass and energy in a system's rest frame, where the two values differ only by a constant and the units of measurement.

The formula defines the energy E of a particle in its rest frame as the product of mass (m) with the speed of light squared ($c^{2}$). Because the speed of light is a large number in everyday units (approximately 300000 km/s or 186000 mi/s), the formula implies that a small amount of rest mass corresponds to an enormous amount of energy.

\vspace{1cm}


\boldmath
\begin{equation}
  E=mc^2
\end{equation}

\vspace{1cm}

Mass–energy equivalence arose from special relativity as a paradox described by the French polymath Henri Poincaré. Einstein was the first to propose the equivalence of mass and energy as a general principle and a consequence of the symmetries of space and time. The principle first appeared in "Does the inertia of a body depend upon its energy-content?", one of his papers, published on 21 November 1905.
\vspace{1cm}

\begin{tabular}{|c|l|}
    \hline
    $E$ & Energy of a particle in its rest frame\\
    $m$ & mass\\
    $c$ &  speed of light\\
    \hline
\end{tabular}


\end{document}
